\documentclass[12pt]{article}
\usepackage[T1]{fontenc}
\usepackage[utf8]{inputenc}
\usepackage{hyperref, microtype, enumitem, titling, newtxtext, newtxmath, tikz}
\usepackage[section]{minted}
\usepackage[scaled=0.85]{beramono}
\usepackage[spanish]{babel}
\usepackage[margin=1in]{geometry}

\usetikzlibrary{shapes, arrows}

\tikzstyle{terminator} = [rectangle, draw, text centered, rounded corners]
\tikzstyle{process} = [rectangle, draw, text centered]
\tikzstyle{junction} = [circle, draw, minimum height=1em]
\tikzstyle{decision} = [diamond, draw, text centered]
\tikzstyle{data} = [trapezium, draw, text centered, trapezium left angle=60,
                    trapezium right angle=120]
\tikzstyle{connector} = [draw, -latex']

\title{Memoria --- Proyecto final\\
\bigskip
\large Informática --- GIM}
\author{Adam Martínez Oussat}
\date{2024-2025}

\begin{document}

\maketitle

\pagenumbering{gobble}

\tableofcontents

\clearpage

\pagenumbering{arabic}

\section*{Notas}
\addcontentsline{toc}{section}{Notas}

Los tipos de las variables empleadas en los bucles \mintinline{cpp}{for} no se
han documentado.

Puesto que algunas funciones implementadas requieren cierta explicación, se ha
valorado añadir un comentario expandido de los incluidos en la documentación de
Doxygen del archivo fuente.

Las partes del código utilizadas para procesar archivos se han excluido del
diagrama de flujo de la función principal.

\section*{Definiciones de tipos}
\addcontentsline{toc}{section}{Definiciones de tipos}

\subsection*{Estructuras}
\addcontentsline{toc}{subsection}{Estructuras}

\subsubsection*{\mintinline{cpp}{Fecha}}
\addcontentsline{toc}{subsubsection}{\mintinline{cpp}{Fecha}}

\begin{itemize}

\item \mintinline{cpp}{unsigned short mes}. El número de meses siempre está en
el rango 1-12. El rango del tipo seleccionado basta.

\item \mintinline{cpp}{unsigned short dia}. El número de días siempre está en el
rango 1-31. El rango del tipo seleccionado basta.

\item \mintinline{cpp}{unsigned short anyo}. El número de años siempre cubre
cifras de 4. El rango del tipo seleccionado basta.

\end{itemize}

\subsubsection*{\mintinline{cpp}{Jugador}}
\addcontentsline{toc}{subsubsection}{\mintinline{cpp}{Jugador}}

\begin{itemize}

\item \mintinline{cpp}{std::string nombre}. Se ha utilizado una cadena porque
debe contener múltiples caracteres del alfabeto.

\item \mintinline{cpp}{Fecha nacimiento}. Se ha empleado la estructura de datos
\mintinline{cpp}{Fecha}, porque debe contener meses, días y años.

\item \mintinline{cpp}{unsigned short jugadas}. Se ha empleado un entero
positivo pequeño, porque el máximo de jugadas siempre será inferior al número de
celdas totales o área del tablero, y ésta asimismo solo puede ser 64 de acuerdo
al documento especificativo.

\end{itemize}

\subsubsection*{\mintinline{cpp}{Estado}}
\addcontentsline{toc}{subsubsection}{\mintinline{cpp}{Estado}}

\begin{itemize}

\item \mintinline{cpp}{unsigned short n_minas}. El máximo número de minas solo
puede ser 8, de acuerdo al documento especificativo, y este rango se puede
cubrir con un entero positivo pequeño.

\item \mintinline{cpp}{bool mina}. Debe comprobar la condición de mina en una
celda, por tanto un valor booleano.

\item \mintinline{cpp}{bool destapada}. Debe comprobar si una mina ha sido
destapada por el usuario o una acción del usuario, con lo que un valor booleano.

\item \mintinline{cpp}{bool bandera}. Debe comprobar si una mina está marcada
con una bandera, por lo que un valor booleano.

\end{itemize}

\subsubsection*{\mintinline{cpp}{Posicion}}
\addcontentsline{toc}{subsubsection}{\mintinline{cpp}{Posicion}}

\begin{itemize}

\item \mintinline{cpp}{unsigned short fil[]}. Almacena valores auxiliares en los
rangos 0-8. El tipo utilizado basta.

\item \mintinline{cpp}{unsigned short col[]}. Almacena valores auxiliares en los
rangos 0-8. El tipo utilizado basta.

\end{itemize}

\subsection*{Constantes}
\addcontentsline{toc}{subsection}{Constantes}

\begin{itemize}

\item \mintinline{cpp}{std::string limpiar}. El comando para limpiar la terminal
del sistema está compuesto de múltiples caracteres. Este tipo es necesario.

\item \mintinline{cpp}{unsigned short MAX_MINAS}. De acuerdo con el documento
especificativo del proyecto, debe ser 8, un rango cubierto por este tipo.

\item \mintinline{cpp}{unsigned short MAX_JUGADORES}. De acuerdo con el
documento especificativo, debe ser 100, un rango cubierto por este tipo.

\item \mintinline{cpp}{unsigned short FIL}. De acuerdo con el documento
especificativo, debe ser 8, un rango cubierto por este tipo.

\item \mintinline{cpp}{unsigned short COL}. De acuerdo con el documento
especificativo, debe ser 8, un rango cubierto por este tipo.

\end{itemize}

\subsection*{Definiciones de tipos compuestos}
\addcontentsline{toc}{subsection}{Definiciones de tipos compuestos}

\begin{itemize}

\item \mintinline{cpp}{Jugador VectorJ[]}. El array debe contener datos de
múltiples jugadores, por lo que se emplea la estructura de
\mintinline{cpp}{Jugador} para cada elemento.

\item \mintinline{cpp}{Estado Tablero[][]}. El tablero debe estar compuesto por
una matriz designada por el número de filas y columnas, y cada elemento debe
tener una descripción particular con todos los campos de la estructura.

\end{itemize}

\subsection*{Funciones}
\addcontentsline{toc}{subsection}{Funciones}

\subsubsection*{\mintinline{cpp}{main()}}
\addcontentsline{toc}{subsubsection}{\mintinline{cpp}{main()}}

\begin{itemize}

\item \mintinline{cpp}{char opcion}. Solo se debe elegir entre un rango de
múltiples letras en el menú. Para ello, caracteres individuales bastan.

\item \mintinline{cpp}{unsigned short fil}. Se utiliza para almacenar el número
de fila obtenido como dato de entrada del usuario. De acuerdo al documento
especificativo, solo se juega con 8 filas, con lo que este tipo basta.

\item \mintinline{cpp}{unsigned short col}. Se utiliza para almacenar el número
de columna obtenido como dato de entrada del usuario. De acuerdo al documento
especificativo, solo se juega con 8 columnas, con lo que este tipo basta.

\item \mintinline{cpp}{unsigned short tam}. El tamaño del array solo ascendará
al máximo establecido en el documento especificativo. Este máximo puede ser
cubierto por el rango del tipo.

\item \mintinline{cpp}{unsigned short intentos}. El número de intentos siempre
será inferior al número de celdas totales. El número de celdas totales será 64,
de acuerdo con el documento especificativo, por lo que el tipo basta.

\item \mintinline{cpp}{std::string f_nom}. Almacena el nombre de archivos a
abrir o crear. De acuerdo con los documentos de ejemplo, los nombres de archivos
están compuestos de múltiples caracteres. Este tipo es necesario.

\item \mintinline{cpp}{std::ifstream f_in}. El flujo de los archivos de entrada
es necesario.

\item \mintinline{cpp}{std::ofstream f_out}. El flujo de los archivos de salida
es necesario.

\item \mintinline{cpp}{Jugador jug}. Se deben almacenar datos del jugador de la
última partida. De acuerdo con el documento especificativo, estos datos se
detallan en la estructura correspondiente.

\item \mintinline{cpp}{VectorJ jugadores}. Los datos de todos los jugadores
deben almacenarse de acuerdo con los datos de los archivos de ejemplo. El
documento especificativo requiere una estructura de jugadores para cada
elemento.

\item \mintinline{cpp}{Tablero tablero}. El tablero debe estar compuesto de
múltiples celdas con datos particulares. Para cada celda, el documento
especificativo detalla los datos de una estructura necesarios.

\end{itemize}

\subsubsection*{\mintinline{cpp}{Menu()}}
\addcontentsline{toc}{subsubsection}{\mintinline{cpp}{Menu()}}

\begin{itemize}

\item \mintinline{cpp}{char opcion}. Solo se debe elegir entre un rango de
múltiples letras en el menú. Para ello, caracteres individuales bastan.

\end{itemize}

\subsubsection*{\mintinline{cpp}{InicializaDesdeFichero()}}
\addcontentsline{toc}{subsubsection}{\mintinline{cpp}{InicializaDesdeFichero()}}

\begin{itemize}

\item \mintinline{cpp}{std::string linea}. La lógica de la función procesa
cada línea del archivo. Para ello, se deben almacenar todos los caracteres de la
línea. Este tipo es necesario.

\end{itemize}

\subsubsection*{\mintinline{cpp}{InicializaAleatoriamente()}}
\addcontentsline{toc}{subsubsection}{%
\mintinline{cpp}{InicializaAleatoriamente()}}

\begin{itemize}

\item \mintinline{cpp}{unsigned short num_minas}. El número máximo de minas solo
será 8, de acuerdo con el documento especificaivo. El rango de este tipo basta.

\item \mintinline{cpp}{unsigned short tmp}. Solo se emplean valores auxiliares
en el rango 0-8, por lo que el tipo empleado basta.

\item \mintinline{cpp}{Posicion minas_pos}. Se utilizan valores binarios de 0s y
1s. Los tipos de la estructura empleada bastan.

\end{itemize}

\subsubsection*{\mintinline{cpp}{NumeroMinasVecinas()}}
\addcontentsline{toc}{subsubsection}{\mintinline{cpp}{NumeroMinasVecinas()}}

\begin{itemize}

\item \mintinline{cpp}{unsigned short minas}. El número de posibles minas
aleatorias está en el rango 0-8. El tipo seleccionado basta.

\item \mintinline{cpp}{Posicion pos}. Se utilizan valores auxiliares en el rango
0-8. El rango del tipo seleccionado basta.

\end{itemize}

\subsubsection*{\mintinline{cpp}{MuestraTablero()}}
\addcontentsline{toc}{subsubsection}{\mintinline{cpp}{MuestraTablero()}}

\begin{itemize}

\item \mintinline{cpp}{unsigned short esp_fil}. El número empleado siempre será
mayor a 1. Según el documento especificativo, el valor siempre será 1. El tipo
selecccionado tiene un rango suficiente.

\item \mintinline{cpp}{unsigned short esp_col}. El número empleado siempre será
mayor a 1. Según el documento especificativo, el valor siempre será 1. El tipo
selecccionado tiene un rango suficiente.

\end{itemize}

\subsubsection*{\mintinline{cpp}{FinJuego()}}
\addcontentsline{toc}{subsubsection}{\mintinline{cpp}{FinJuego()}}

\begin{itemize}

\item \mintinline{cpp}{bool mina}. Recoge la condición de uno de los posibles
finales del juego. Esto requiere un valor booleano.

\item \mintinline{cpp}{bool proc}. Recoge la condición de uno de los posibles
finales del juego. Esto requiere un valor booleano.

\item \mintinline{cpp}{bool fin}. Recoge la condición del final del juego. Esto
requiere un valor booleano.

\end{itemize}

\subsubsection*{\mintinline{cpp}{MinaAbierta()}}
\addcontentsline{toc}{subsubsection}{\mintinline{cpp}{MinaAbierta()}}

\begin{itemize}

\item \mintinline{cpp}{bool abierta}. Recoge la condición de encontrar una
celda-mina abierta. Esto requiere un valor booleano.

\end{itemize}

\subsubsection*{\mintinline{cpp}{TodasCeldasProcesadas()}}
\addcontentsline{toc}{subsubsection}{\mintinline{cpp}{TodasCeldasProcesadas()}}

\begin{itemize}

\item \mintinline{cpp}{bool procesadas}. Recoge la condición de haber procesado
todas las celdas del tablero. Esto requiere un valor booleano.

\item \mintinline{cpp}{unsigned short num_procesadas}. El número de celdas
totales tan solo asciende a 64 de acuerdo con el documento especificativo. El
número de celdas procesadas solo será menor o igual a 64. El rango del tipo
seleccionado basta.

\end{itemize}

\subsubsection*{\mintinline{cpp}{LeeJugadoresFichero()}}
\addcontentsline{toc}{subsubsection}{\mintinline{cpp}{LeeJugadoresFichero()}}

\begin{itemize}

\item \mintinline{cpp}{std::string linea}. La lógica de la función procesa cada
línea del archivo. Para ello, se deben almacenar todos los caracteres de la
línea. Este tipo es necesario.

\end{itemize}

\subsubsection*{\mintinline{cpp}{LeeInfoJugador()}}
\addcontentsline{toc}{subsubsection}{\mintinline{cpp}{LeeInfoJugador()}}

\begin{itemize}

\item \mintinline{cpp}{Jugador jug}. Se deben almacenar datos del jugador de la
última partida. De acuerdo al documento especificativo, estos datos se detallan
en la estructura correspondiente.

\end{itemize}

\subsubsection*{\mintinline{cpp}{InsertaJugadorVector()}}
\addcontentsline{toc}{subsubsection}{\mintinline{cpp}{InsertaJugadorVector()}}

\begin{itemize}

\item \mintinline{cpp}{bool mem}. Evalúa si el array de jugadores está lleno.
Esto requiere un valor booleano.

\end{itemize}

\clearpage

\section*{Implementación de los subprogramas}
\addcontentsline{toc}{section}{Implementación de los subprogramas}

\subsection*{\mintinline{cpp}{char Menu()}}
\addcontentsline{toc}{subsection}{\mintinline{cpp}{char Menu()}}

\begin{description}

\item[\emph{Propósito}]

Devuelve la opción elegida entre las disponibles en un menú impreso por
pantalla. Para cada opción introducida por el usuario, se comprueba si ésta se
encuentra en el rango de posibles opciones.

El menú ofrece las siguientes opciones:

\begin{itemize}

\item Leer las posiciones de las minas de un archivo con un formato fijo e
inicializar el resto del tablero.
\item Generar semialeatoriamente las posiciones de las minas e inicializar el
resto del tablero.
\item Descubrir una celda de un tablero ya inicializado.
\item Añadir una bandera a una celda específica del tablero.
\item Desmarcar celdas de ser banderas si se han marcado como banderas.

\end{itemize}

\item[\emph{Parámetros}]

Ninguno.

\end{description}

\subsection*{\mintinline{cpp}{void InicializaDesdeFichero()}}
\addcontentsline{toc}{subsection}{%
\mintinline{cpp}{void InicializaDesdeFichero()}}

\begin{description}

\item[\emph{Propósito}]

A partir de un archivo de entrada previamente abierto, se lee del mismo la
información relativa a las posiciones de las minas. Estos datos se almacenan en
el array correspondiente al tablero. La función también inicializa todas las
celdas sin minas a valores por defecto.

El archivo debe contener posiciones válidas, y el número de minas máximo debe
ajustarse al establecido en el documento especificativo. No se realiza ninguna
comprobación de ello en la función.

\item[\emph{Parámetros}] \leavevmode

\begin{description}

\item[Salida] \mintinline{cpp}{tablero}. Array con los estados de cada celda del
tablero. Aunque no es explícito, se pasa por referencia porque es un array y
éste es el caso en C++. Es un parámetro de salida porque solo se modifican sus
valores, y no se obtiene ninguna información del mismo.

\item[Entrada] \mintinline{cpp}{f}. Flujo del archivo con los datos a leer. El
único modo de pasar flujos de archivo es por referencia. A pesar de ello, es un
parámetro de entrada porque solo brinda información y no se cambia de ningún
modo.

\end{description}

\end{description}

\subsection*{\mintinline{cpp}{void InicializaAleatoriamente()}}
\addcontentsline{toc}{subsection}{%
\mintinline{cpp}{void InicializaAleatoriamente()}}

\begin{description}

\item[\emph{Propósito}]

Inicializa los estados de cada una de las celdas del tablero con valores por
defecto, a excepción de ciertas celdas. Las excepciones, elegidas de manera
semideterminística, son celdas donde se colocarán minas. El número de minas a
colocar varía en el rango 1-8, y se elige también de forma semialeatoria.

La función asigna un valor semialeatorio al número de minas hasta conseguir un
valor superior a 0. Tras ello, inicializa unas posiciones auxiliares de minas
para marcar ciertas filas y columnas con un valor de minas. Finalmente, itera
entre todas las celdas del tablero para modificar aquellas que coincidan con las
mismas casillas auxiliares. Puesto que la implementación hace uso de arrays
auxiliares diferentes para cada posición de mina, se requiere modificar los
valores binarios de las posiciones auxiliares tras encontrar cada \emph{match}
relativo a las posiciones reales del tablero. De otro modo, columnas completas
se verían marcadas como minas.

\item[\emph{Parámetros}] \leavevmode

\begin{description}

\item[Salida] \mintinline{cpp}{tablero}. Array de filas y columnas a inicializar
con valores semideterminísticos para las posiciones con minas y por defecto para
el resto de posiciones. Es un parámetro de salida porque solo se escriben en el
mismo los valores a inicializar; los valores anteriores son basura o relativos a
una inicialización previa. Se pasa por referencia ``automáticamente'' porque es
necesario modificar los valores en memoria.

\end{description}

\end{description}

\subsection*{\mintinline{cpp}{unsigned short NumeroMinasVecinas()}}
\addcontentsline{toc}{subsection}{%
\mintinline{cpp}{unsigned short NumeroMinasVecinas()}}

\begin{description}

\item[\emph{Propósito}]

Devuelve el número de celdas vecinas con la propiedad de minas verdadera. Si el
número de minas alrededor es 0, se devuelve también el valor 0.

La función implementa una selección de rangos de celdas a explorar relativo a la
posición de la celda inicial proveída. Esto se debe a la necesidad de evitar
explorar índices de celdas fuera del rango alrededor de la celda inicial.

\item[\emph{Parámetros}] \leavevmode

\begin{description}

\item[Entrada] \mintinline{cpp}{tablero}. Array con las filas, columnas y
estados de cada una de las celdas. Se emplea solo para la lectura de la
propiedad de mina. Es un parámetro de entrada y se pasa por valor porque no es
necesario modificar su valor en memoria en el ámbito de la llamada.

\item[Entrada] \mintinline{cpp}{fil}. Fila inicial de la exploración. Solo se
precisa acceso de lectura relativo al valor en memoria desde la llamada para
leer si las correspondientes celdas contienen minas. Es por tanto un parámetro
de entrada pasado por valor.

\item[Entrada] \mintinline{cpp}{col}. Columna inicial de la exploración. Solo se
precisa acceso de lectura relativo al valor en memoria desde la llamada para
leer si las correspondientes celdas contienen minas. Es por tanto un parámetro
de entrada pasado por valor.

\end{description}

\end{description}

\subsection*{\mintinline{cpp}{void MuestraTablero()}}
\addcontentsline{toc}{subsection}{\mintinline{cpp}{void MuestraTablero()}}

\begin{description}

\item[\emph{Propósito}]

Imprime por pantalla la situación presente del tablero de valores para el juego.
Si está inicializado, presentará los valores adecuados, pero si no lo está,
interpretará cualquier valor en memoria asignado a los campos de registros de
cada uno de los elementos en el array.

La función implementa la capacidad para mostrar cualquier tipo de tablero,
independientemente de si conforma con la matriz cuadradada de 8x8 del documento
especificativo. La implementación recorre los valores de filas y columnas,
imprimiendo también índices para cada fila y columna procesada. En los detalles
del proyecto no se emplea ningún tipo de índice natural, pero para esta
implementación se hace uso de índices comenzados en 1. Esto viene con algunos
requerimientos adicionales en otras funciones comentados al cabo del resto de la
memoria.

Al recorrer cada fila, se realiza un cómputo de espacios siempre presentes en la
impresión de caracteres, y tras ello se cuentan los caracteres del número
máximo de filas para añadir espacios adicionales (1 solo si se tratase de un
número de filas con un solo caracter, 2 si se tratase de un número de filas con
2 caracteres, etc.) Se hace uso del mismo tipo de funcionamiento para procesar
espacios en los índices de columnas. Se tomó la decisión de abstraer el cálculo
de caracteres en variables específicas porque este procesamiento se requiere en
múltiples ocasiones.

\item[\emph{Parámetros}] \leavevmode

\begin{description}

\item[Entrada] \mintinline{cpp}{tablero}. Array con las filas y columnas del
tablero. Solo es necesario leer datos del mismo. Por tanto, no se precisa pasar
el puntero al primer elemento, porque cualquier operación no intencionada podría
considerarse tener acceso al puntero dereferenciado. Se pasa por valor porque es
necesario para evitar modificar valores en memoria relativos a la región donde
se llamó la función. Por ello, también es un parámetro de entrada.

\end{description}

\end{description}

\subsection*{\mintinline{cpp}{void LeeCelda()}}
\addcontentsline{toc}{subsection}{\mintinline{cpp}{void LeeCelda()}}

\begin{description}

\item[\emph{Propósito}]

Procesa la lectura de celdas desde teclado por el usuario. Cabe mencionar la
modificación de valores de filas y columnas de los proveídos por el usuario.
Esto es parte de un intento por simplificar el modo de presentar los índices, de
manera natural, e igualmente ejecutar cualquier función del programa con los
índices correctos de los arrays del lenguaje.

\item[\emph{Parámetros}] \leavevmode

\begin{description}

\item[Salida] \mintinline{cpp}{fil}. Número de la fila a modificar con el valor
correspondiente. Se pasa por referencia porque una copia del puntero es
necesaria para modificar el valor dereferenciado de la dirección. Es de salida
porque se requiere escribir en la región de memoria correspondiente, y no es
necesario leer los valores almacenados.

\item[Salida] \mintinline{cpp}{col}. Número de la columna a modificar con el
valor correspondiente. Se pasa por referencia porque una copia del puntero es
necesaria para modificar el valor dereferenciado de la dirección. Es de salida
porque se requiere escribir en la región de memoria correspondiente, y no es
necesario leer los valores almacenados.

\end{description}

\end{description}

\subsection*{\mintinline{cpp}{void AbreCelda()}}
\addcontentsline{toc}{subsection}{\mintinline{cpp}{void AbreCelda()}}

\begin{description}

\item[\emph{Propósito}]

Modifica el estado de las posiciones del tablero. Para ello, abre la celda
pasada, y si aquellas a su alrededor cumplen tres requerimientos, las explora
recursivamente del mismo modo. Los tres requerimientos son:

\begin{itemize}

\item Pasar la comprobación sobre la existencia de tal posición. Ésto se
implementa de acuerdo al esquema organizativo mostrado en el documento
especificativo.
\item El número de minas vecinas es 0. La implementación mostrada en vídeo como
muestra del proyecto no pasa esta comprobación, y por tanto revela todas las
celdas sin minas. Se ha valorado demasiado sencillo el juego de esa manera, y
por tanto solo se mostrarán celdas sin contacto con otras celdas-minas (aquellas
sin indicador numérico de mina.)
\item La celda está destapada. Ésto es necesario como marcador de aquellas
celdas ya exploradas. Gracias a ello, la recursión no es infinita entre la
primera celda explorada y el primer \emph{match}, dependiente de la posición de
la misma.

\end{itemize}

\item[\emph{Parámetros}] \leavevmode

\begin{description}

\item[Entrada, salida] \mintinline{cpp}{tablero}. Array con la información sobre
filas y columnas a modificar. La función requiere leerla, y dependiendo de las
celdas exploradas, modificar la información de una o más celdas. Se pasa por
referencia porque es necesario modificar los valores en memoria.

\item[Entrada] \mintinline{cpp}{fil}. Número de la fila correspondiente a la
celda a explorar. Su valor solo es necesario para tener información sobre la
celda a destapar, y si califica, las celdas a explorar. Por tanto, se pasa por
valor para obtener una copia del valor en memoria y proteger el valor del ámbito
donde se ha realizado la llamada a la función.

\item[Entrada] \mintinline{cpp}{col}. Número de la columna correspondiente a la
celda a explorar. Su valor solo es necesario para tener información sobre la
celda a destapar, y si califica, las celdas a explorar. Por tanto, se pasa por
valor para obtener una copia del valor en memoria y proteger el valor del ámbito
donde se ha realizado la llamada a la función.

\end{description}

\end{description}

\subsection*{\mintinline{cpp}{bool FinJuego()}}
\addcontentsline{toc}{subsection}{\mintinline{cpp}{bool FinJuego()}}

\begin{description}

\item[\emph{Propósito}]

Comprueba si la partida ha finalizado a partir de los dos escenarios indicados
en el documento especificativo. Estos escenarios consisten en encontrar una mina
destapada o haber procesado todas las celdas del tablero. La comprobación se
devuelve en la forma de un valor booleano. De acuerdo con las recomendaciones
del documento especificativo, se han separado estas posibilidades; ello ha
probado especialmente útil a la hora de implementar la lógica del final del
juego.

\item[\emph{Parámetros}] \leavevmode

\begin{description}

\item[Entrada] \mintinline{cpp}{tablero}. Array con las filas y columnas a leer,
para verificar alguna de las condiciones del final del juego. Solo se espera
leer los estados de cada celda del tablero. Por tanto, únicamente se precisa una
copia del valor.

\end{description}

\end{description}

\subsection*{\mintinline{cpp}{bool MinaAbierta()}}
\addcontentsline{toc}{subsection}{\mintinline{cpp}{bool MinaAbierta()}}

\begin{description}

\item[\emph{Propósito}]

Comprueba si alguna de las celdas del tablero conteniendo una mina ha sido
destapada. Ésto es uno de los posibles escenarios para determinar el final del
juego. Devolverá un valor booleano dependiendo de si se ha encontrado una mina
abierta.

\item[\emph{Parámetros}] \leavevmode

\begin{description}

\item[Entrada] \mintinline{cpp}{tablero}. Array con las filas y columnas a leer,
para verificar la condición. Solo se espera leer los estados de cada celda del
tablero. Por tanto, únicamente se precisa de un valor de copia del array.

\end{description}

\end{description}

\subsection*{\mintinline{cpp}{bool TodasCeldasProcesadas()}}
\addcontentsline{toc}{subsection}{%
\mintinline{cpp}{bool TodasCeldasProcesadas()}}

\begin{description}

\item[\emph{Propósito}]

Comprueba si todas las celdas del tablero han sido procesadas. Para la
implementación de esta condición de victoria, a falta de una definición formal
para el proyecto, se ha decidido comprobar si todas las celdas sin minas han
sido destapadas, y si todas las celdas con minas tienen una bandera. La primera
condición es exclusiva de las celdas sin minas, pero la lógica del programa ya
implementa una comprobación de minas previa. Por tanto, se considera innecesaria
la comprobación de minas en dicha ocasión. Devuelve un valor booleano
dependiente de la equivalencia entre las celdas procesadas y las celdas totales
del tablero (64 de acuerdo al documento especificativo.)

\item[\emph{Parámetros}] \leavevmode

\begin{description}

\item[Entrada] \mintinline{cpp}{tablero}. Array con las filas y columnas a
considerar al cabo del recorrido por el tablero. Solo requiere leer los estados
de las celdas. Por ese motivo es de entrada y se pasa la referencia como
constante, para evitar modificaciones al valor en memoria al que se apunta.

\end{description}

\end{description}

\subsection*{\mintinline{cpp}{void LeeJugadoresFichero()}}
\addcontentsline{toc}{subsection}{\mintinline{cpp}{void LeeJugadoresFichero()}}

\begin{description}

\item[\emph{Propósito}]

Lee desde el archivo los datos de múltiples jugadores, y los almacena en un
array de jugadores. Esta función forma parte del final del juego, y emplea
también una variable para determinar el número de espacios ocupados por los
jugadores almacenados. Ésto es necesario debido al uso de arrays de tamaño
``fijo'' en C++.

Como el resto de funciones operando en archivos, el archivo debe estar abierto
de antemano, y ya se debe haber comprobado sus \mintinline{cpp}{badbit},
\mintinline{cpp}{eofbit} y \mintinline{cpp}{failbit}.

La implementación recoge cada línea del archivo, y aprovecha el formato estándar
proveído para los dos ejemplos de jugadores. Los nombres corresponden con las
líneas conteniendo un caracter alfabético. La fecha de nacimiento corresponderá
a la línea con espacios para separar días, y cualquier otro caso solo puede
tratarse de las jugadas.

\item[\emph{Parámetros}] \leavevmode

\begin{description}

\item[Salida] \mintinline{cpp}{jugadores}. Array de jugadores con los datos
leídos relativos a cada jugador. Esta información solo se escribirá en el array;
cualquier otro valor almacenado en el mismo previamente es basura. Por este
motivo, es un parámetro exclusivamente de salida y se pasa por referencia al
valor de la dirección en memoria.

\item[Salida] \mintinline{cpp}{tam}. Tamaño del array rellenado con los
jugadores en fichero. Este ya se asume inicializado previo a la llamada. Por
ello, el puntero copiado directamente será modificado con operaciones sobre sí
mismo.

\item[Entrada] \mintinline{cpp}{f}. Archivo con los datos de los jugadores a
leer. Es necesario pasarlo por referencia porque se trata de un flujo de
archivo, a pesar de únicamente emplear su información y no modificarlo de ningún
modo. Solo se lee información del mismo, con lo que es un parámetro de entrada.

\end{description}

\end{description}

\subsection*{\mintinline{cpp}{Jugador LeeInfoJugador()}}
\addcontentsline{toc}{subsection}{\mintinline{cpp}{Jugador LeeInfoJugador()}}

\begin{description}

\item[\emph{Propósito}]

Recoge información sobre un jugador por teclado, y la almacena en una estructura
de jugador. Ésta conforma también el valor de retorno de la función. Los datos
son relativos a la partida jugada. Por este motivo, funciona como un guardado de
partida y no sirve para modificar registros existentes.

\item[\emph{Parámetros}] \leavevmode

\begin{description}

\item[Entrada] \mintinline{cpp}{intentos}. Los intentos a registrar realizados
durante la partida. Se pasa como parámetro de entrada y por valor porque solo se
espera leer su valor en memoria. Refiere al evento de descubrir una celda.

\end{description}

\end{description}

\subsection*{\mintinline{cpp}{bool InsertaJugadorVector()}}
\addcontentsline{toc}{subsection}{\mintinline{cpp}{bool InsertaJugadorVector()}}

\begin{description}

\item[\emph{Propósito}]

Inserta un jugador en memoria para añadir los datos de la última partida a los
datos existentes. No es posible añadir valores de jugadores ya existentes, y
tampoco se pueden modificar valores registrados de los archivos; la
implementación siempre añadirá un nuevo registro de jugador. La función también
comprueba si el array de jugadores está lleno y la devuelve como valor de
retorno. Ésto se debe al limitado rango de 100 jugadores dado por la
correspondiente constante.

\item[\emph{Parámetros}] \leavevmode

\begin{description}

\item[Entrada] \mintinline{cpp}{jug}. Estructura con los datos del jugador a
insertar. Se espera solo leer los datos de la estructura; por ello, se pasa una
copia del valor en memoria, y es de entrada.

\item[Salida] \mintinline{cpp}{jugadores}. Array con la información de todos los
jugadores, incluidos aquellos en el archivo leído en otra parte del programa. Es
necesario escribir un nuevo elemento; por ello, se requiere acceso al valor en
memoria donde se realizó la llamada, y se pasa por referencia. Asimismo, es un
parámetro de salida porque solo se requiere escribir en el mismo.

\item[Entrada, salida] \mintinline{cpp}{tam}. La cantidad de elementos no-basura
del array de jugadores, determinado previamente con un archivo con jugadores, y
presentemente al añadir un jugador. Su valor se utiliza tanto para añadir el
jugador al final del array, como para incrementar su tamaño tras realizar esta
operación. El valor en memoria de la llamada se desea modificar para propósitos
finales sobre el array de jugadores. Por este motivo, se pasa por referencia el
puntero a la dirección del valor en el ámbito de la llamada.

\end{description}

\end{description}

\subsection*{\mintinline{cpp}{void EscribeJugadoresFichero()}}
\addcontentsline{toc}{subsection}{%
\mintinline{cpp}{void EscribeJugadoresFichero()}}

\begin{description}

\item[\emph{Propósito}]

Escribe datos de un array de jugadores en un archivo de salida. Puede ser
cualquier archivo; incluyendo el archivo con jugadores ya existentes, o uno
nuevo.

\item[\emph{Parámetros}] \leavevmode

\begin{description}

\item[Entrada] \mintinline{cpp}{jugadores}. Array de jugadores con múltiples
registros. Solo se deben escribir los datos del array en el archivo de salida.
Por esta razón, solo es necesario pasar una copia por valor. Esto también
provoca el tratamiento como parámetro de entrada; no se escribirán valores, solo
se leerán.

\item[Entrada] \mintinline{cpp}{tam}. Espacios ocupados en el array; para evitar
asignar valores basura de posiciones sin valores inicializados, es necesario
recorrer el array solo hasta la posición debida. Por ello, no se precisa
modificar el valor en memoria en el ámbito de la llamada, y se pasa por valor;
ello también lo hace un parámetro de entrada.

\item[Salida] \mintinline{cpp}{f}. Archivo de salida donde escribir los datos
del array de jugadores. Se pasa por referencia porque ésto es necesario con
flujos de archivo en C++, y es un parámetro de salida porque solo se escribe en
el mismo.

\end{description}

\end{description}

\clearpage

\section*{Diagrama de flujo de la función \mintinline{cpp}{main()}}
\addcontentsline{toc}{section}{Diagrama de flujo de la función
\mintinline{cpp}{main()}}

\begin{center}

\begin{tikzpicture}

\node [terminator] at (0,0) (start) {\textbf{Inicio}};
\node [process] at (0,-1) (size) {\emph{tam} $\gets$ 0};
\node [process] at (0,-2) (attempts) {\emph{intentos} $\gets$ 0};
\node [junction] at (0,-3) (junct1) {};
\node [process] at (0,-4) (option) {\emph{opción} $\gets$ Opción de menu};
\node [decision] at (0,-6) (switch) {\textbf{¿\emph{opción}?}};
\node [junction] at (-6.5,-8) (junct2) {};
\node [data] at (0,-8) (a1) {Leer \emph{archivo minas}};
\node [process] at (0,-9) (a2) {Inicializar tablero desde \emph{archivo
minas}};
\node [junction] at (-6.5,-10) (junct4) {};
\node [process] at (0,-10) (b) {Inicializar tablero aleatoriamente};
\node [junction] at (-6.5,-11) (junct5) {};
\node [data] at (0,-11) (c1) {Leer \emph{celda}};
\node [decision] at (0,-14) (c2) {\textbf{¿\emph{celda} destapada?}};
\node [process] at (-4,-14) (c3) {Abrir \emph{celda}};
\node [process] at (-4,-13) (c4) {\emph{intentos} $\gets$ \emph{intentos} + 1};
\node [junction] at (-6.5,-17) (junct6) {};
\node [data] at (0,-17) (d1) {Leer \emph{celda}};
\node [process] at (0,-18) (d2) {Marcar \emph{celda}};
\node [data] at (0,-19) (e1) {Leer \emph{celda}};
\node [decision] at (8,-17.5) (e2) {\textbf{¿\emph{celda} marcada?}};
\node [process] at (3,-16) (e3) {Desmarcar \emph{celda}};
\node [junction] at (8,-14.5) (junct3) {};
\node [junction] at (8,-9) (junct7) {};
\node [junction] at (8,-13.5) (junct8) {};
\node [junction] at (3,-11.5) (junct9) {};
\node [junction] at (8,-11.5) (junct10) {};
\node [junction] at (8,-10) (junct11) {};
\node [process] at (8,-8) (show) {Mostrar tablero};
\node [decision] at (8,-5.5) (fin) {\textbf{¿Fin juego?}};

\path [connector] (start) -- (size);
\path [connector] (size) -- (attempts);
\path [connector] (attempts) -- (junct1);
\path [connector] (junct1) -- (option);
\path [connector] (option) -- (switch);
\path [connector] (switch) -| (junct2);
\path [connector] (junct2) -- node[anchor=south] {`a'} (a1);
\path [connector] (junct2) -- (junct4);
\path [connector] (junct4) -- node[anchor=south] {`b'} (b);
\path [connector] (junct4) -- (junct5);
\path [connector] (junct5) -- node[anchor=south] {`c'} (c1);
\path [connector] (junct5) -- (junct6);
\path [connector] (junct6) -- node[anchor=south] {`d'} (d1);
\path [connector] (junct6) -- (-6.5,-19) -- node[anchor=south] {`e'} (e1);
\path [connector] (a1) -- (a2);
\path [connector] (a2) -- (junct7);
\path [connector] (b) -- (junct11);
\path [connector] (c1) -- (c2);
\path [connector] (c2) -- node[anchor=north] {V} (3,-14) -- (junct9);
\path [connector] (c2) -- node[anchor=north] {F} (c3);
\path [connector] (c3) -- (c4);
\path [connector] (c4) |- (junct9);
\path [connector] (junct9) -- (junct10);
\path [connector] (d1) -- (d2);
\path [connector] (d2) -| (5.5,-13.5) -- (junct8);
\path [connector] (e1) |- (3,-20) -| (e2);
\path [connector] (e2) -- node[anchor=south] {V} (3,-17.5) -- (e3);
\path [connector] (e2) -- node[anchor=east] {F} (junct3);
\path [connector] (e3) |- (junct3);
\path [connector] (junct3) -- (junct8);
\path [connector] (junct8) -- (junct10);
\path [connector] (junct10) -- (junct11);
\path [connector] (junct11) -- (junct7);
\path [connector] (junct7) -- (show);
\path [connector] (show) -- (fin);
\path [connector] (fin) -- node[anchor=north] {V\footnotemark} (5,-5.5);
\path [connector] (fin) -- node[anchor=east] {F} (8,-3) -- (junct1);

\end{tikzpicture}

\footnotetext{Continuado en siguiente página.}

\newpage

\begin{tikzpicture}

\node [decision] at (0,-2.5) (mine1) {\textbf{¿Mina abierta?}};
\node [data] at (-4.5,-2.5) (mine2) {Escribir `Derrota'};
\node [data] at (4.5,-2.5) (mine3) {Escribir `Victoria'};
\node [junction] at (0,-5) (junct1) {};
\node [data] at (0,-6) (attempts) {Escribir \emph{intentos}};
\node [decision] at (0,-9) (proc1) {\textbf{¿Celdas procesadas?}};
\node [data] at (6,-9) (proc2) {Leer \emph{archivo jugadores}};
\node [process] at (6,-10) (proc3) {Procesar jugadores en \emph{archivo
jugadores}};
\node [data] at (6,-11) (proc4) {Leer \emph{info jugador}};
\node [process] at (6,-12) (proc5) {\emph{jug} $\gets$ \emph{info jugador}};
\node [decision] at (6,-15.5) (proc6) {\textbf{¿Memoria suficiente?}};
\node [process] at (0,-15.5) (proc7) {Añadir \emph{jug} a jugadores};
\node [data] at (6,-19) (proc8) {Escribir `Memoria insuficiente'};
\node [junction] at (0,-20) (junct2) {};
\node [data] at (0,-21) (proc9) {Leer \emph{archivo de guardado}};
\node [process] at (0,-22) (proc10) {Añadir jugadores a \emph{archivo de
guardado}};
\node [terminator] at (0,-23) (fin) {\textbf{Fin}};

\path [connector] (0,0) -- (mine1);
\path [connector] (mine1) -- node[anchor=north] {V} (mine2);
\path [connector] (mine1) -- node[anchor=north] {F} (mine3);
\path [connector] (mine2) |- (junct1);
\path [connector] (mine3) |- (junct1);
\path [connector] (junct1) -- (attempts);
\path [connector] (attempts) -- (proc1);
\path [connector] (proc1) -- node[anchor=north] {F} (-4,-9) |- (-4,-23) --
(fin);
\path [connector] (proc1) -- node[anchor=north] {V} (proc2);
\path [connector] (proc2) -- (proc3);
\path [connector] (proc3) -- (proc4);
\path [connector] (proc4) -- (proc5);
\path [connector] (proc5) -- (proc6);
\path [connector] (proc6) -- node[anchor=north] {V} (proc7);
\path [connector] (proc6) -- node[anchor=east] {F} (proc8);
\path [connector] (proc7) -- (junct2);
\path [connector] (proc8) |- (junct2);
\path [connector] (junct2) -- (proc9);
\path [connector] (proc9) -- (proc10);
\path [connector] (proc10) -- (fin);

\end{tikzpicture}

\end{center}

\end{document}
